 %%%%%%%%%%%%%%%%%%%%%%%%%%%%%%%%%%%%%%%%%
% University Assignment Title Page 
% LaTeX Template
% Version 1.0 (27/12/12)
%
% This template has been downloaded from:
% http://www.LaTeXTemplates.com
%
% Original author:
% WikiBooks (http://en.wikibooks.org/wiki/LaTeX/Title_Creation)
%
% License:
% CC BY-NC-SA 3.0 (http://creativecommons.org/licenses/by-nc-sa/3.0/)
% 
% Instructions for using this template:
% This title page is capable of being compiled as is. This is not useful for 
% including it in another document. To do this, you have two options: 
%
% 1) Copy/paste everything between \begin{document} and \end{document} 
% starting at \begin{titlepage} and paste this into another LaTeX file where you 
% want your title page.
% OR
% 2) Remove everything outside the \begin{titlepage} and \end{titlepage} and 
% move this file to the same directory as the LaTeX file you wish to add it to. 
% Then add \input{./title_page_1.tex} to your LaTeX file where you want your
% title page.
%
%%%%%%%%%%%%%%%%%%%%%%%%%%%%%%%%%%%%%%%%%
%\title{Title page with logo}
%----------------------------------------------------------------------------------------
%   PACKAGES AND OTHER DOCUMENT CONFIGURATIONS
%----------------------------------------------------------------------------------------

\documentclass[12pt]{article}
\usepackage[english]{babel}
\usepackage[utf8x]{inputenc}
\usepackage[TS1,T1]{fontenc}
\usepackage{fourier, heuristica}
\usepackage{array, booktabs}
\usepackage{graphicx}
\usepackage[x11names]{xcolor}
\usepackage{colortbl}
\usepackage{caption}
\DeclareCaptionFont{blue}{\color{LightSteelBlue3}}

\newcommand{\foo}{\color{LightSteelBlue3}\makebox[0pt]{\textbullet}\hskip-0.5pt\vrule width 1pt\hspace{\labelsep}}


\begin{document}

\begin{titlepage}

\newcommand{\HRule}{\rule{\linewidth}{0.5mm}} % Defines a new command for the horizontal lines, change thickness here

\center % Center everything on the page
 
%----------------------------------------------------------------------------------------
%   HEADING SECTIONS
%----------------------------------------------------------------------------------------

\textsc{\LARGE Universität Hamburg}\\[1.5cm] % Name of your university/college
\textsc{\Large Programmiertechnisches Praktikum}\\[0.5cm] % Major heading such as course name
%\textsc{\large PTP}\\[0.5cm] % Minor heading such as course title

%----------------------------------------------------------------------------------------
%   TITLE SECTION
%----------------------------------------------------------------------------------------

\HRule \\[0.4cm]
{ \huge \bfseries Flügel der Freiheit}\\[0.4cm] % Title of your document
\HRule \\[1.5cm]
 
%----------------------------------------------------------------------------------------
%   AUTHOR SECTION
%----------------------------------------------------------------------------------------

\begin{minipage}{0.4\textwidth}
\begin{flushleft} \large
\emph{Author:}\\
Erik Nils Knopp\\
Fin Töter\\ 
\end{flushleft}
\end{minipage}
~
\begin{minipage}{0.4\textwidth}
\begin{flushright} \large
\emph{Supervisor:} \\
Dipl.-Inform. Andreas Heymann 
\end{flushright}
\end{minipage}\\[2cm]

% If you don't want a supervisor, uncomment the two lines below and remove the section above
%\Large \emph{Author:}\\
%John \textsc{Smith}\\[3cm] % Your name

%----------------------------------------------------------------------------------------
%   DATE SECTION
%----------------------------------------------------------------------------------------

{\large \today}\\[2cm] % Date, change the \today to a set date if you want to be precise

%----------------------------------------------------------------------------------------
%   LOGO SECTION
%----------------------------------------------------------------------------------------

\includegraphics[width=0.88\textwidth]{logo.jpg}\\ % Include a department/university logo - this will require the graphicx package
 
%----------------------------------------------------------------------------------------

\vfill % Fill the rest of the page with whitespace

\end{titlepage}

\section{Einführung}

Im Rahmen der Veranstaltung "Programmiertechnisches Praktikum" ist es Bestandteil des zweiten Teils ein Projekt in der Sprache Java zu programmieren. Dieses Projekt ist frei wählbar und kann alles beinhalten, die einzige Bedingung war, dass es modular Erweiterbar ist. Im Nachfolgenden werden wir ihnen nun unser Projekt "Flügel der Freiheit" vorstellen. 

\section{Projekt}
\subsection{Einführung}
"Flügel der Freiheit" ist ein simples Arcade Shoot 'em up im Stile von Galaga, Space Invaders etc. Hierbei gibt es  ein Spieler, welcher auf dem Bild frei bewegbar ist, außerdem ist es ihm möglich verschiedene Geschosse abzuschießen, weitergehend kommen aus verschiedenen Richtungen Gegner, welche von dem Spieler abgeschossen werden müssen. Aus der Anzahl der abgeschossenen Gegner in einem Spieldurchlauf berechnet sich dann eine Score. Dies ist modular erweiterbar durch verschiedene Power Ups, verschiedene Gegnertypen, Levelmodifikationen und weiteres. 
\subsection{Programmiertechnische Implementierung}
Wir werden ein rudimentäres 2D Game Framework implementieren, welches Funktionen wie den Render und Dinge wie Kollisionserkennung übernimmt, auf dessen Basis werden wir dann unser Spiel aufbauen mit Hilfe eines Entity-Component Ansatzes. Wie genau sich dies auswirkt werden wir im Laufe der Entwicklung sehen. 
\subsection{Code Style}
Als Code Style wird der offizielle Google Java Style verwendet.

\end{document}